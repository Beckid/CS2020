%-----------------------------------------------------------------------------
%	PACKAGES AND OTHER DOCUMENT CONFIGURATIONS
%-----------------------------------------------------------------------------

\documentclass[12pt]{article} % Default font size is 12pt

\usepackage{geometry} % Required to change the page size to A4
\geometry{a4paper} % Set the page size to be A4
\geometry{left=2.2cm, right=2.2cm, top=3cm, bottom=3cm} % Set bolder sizes

\usepackage{float} % Allows putting an [H] in \begin{figure} to specify the exact location of the figure
\usepackage{wrapfig} % Allows in-line images such as the example fish picture

\usepackage{lipsum} % Used for inserting dummy 'Lorem ipsum' text into the template

\usepackage{indentfirst}

\linespread{1.2} % Line spacing

\setlength{\parindent}{2em} %indentation before each paragraph

%\setlength\parindent{0pt} % Uncomment to remove all indentation from paragraphs

%---------------------------------------------------------------------------
%	Document Information
%---------------------------------------------------------------------------
\begin{document}

\title{0 - 1 Knapsack Problem}
\author{Niu Yunpeng, Wang Junming}
\date{today}

%---------------------------------------------------------------------------
%	Title
%---------------------------------------------------------------------------
\begin{center}

{\large\bf
0 - 1 Knapsack Problem\\
}
\vspace{0.2cm}
\emph{\textbf{Niu Yunpeng A0162492A}}
\end{center}

%---------------------------------------------------------------------------
%	Body Content
%---------------------------------------------------------------------------

\section{Problem}
Given \emph{n} items, each has a weight and a value; and a bag that can hold up to \emph{W kg}. How do you put items in the bag to maximize the total value, but does not exceed weight \emph{W}. Return this value.

%------------------------------------------------
\section{Recurrence Function}
1. Relationship\par
\begin{center}
$V[n, w] = max(V[n - 1, w], value[n] + V[n - 1, w - weight[n]])$\par
\end{center}\par
2. Explanation\par
Assume that there are n items, and we traverse them one by one as [1,..,n]. For each item, we have only two choices, to \emph{put it in the bag} or \emph{not put it in the bag}. So let's consider these two situations seperately:\par
To make things easier, we consider the last item first, that is, the item with index \emph{n}, and with \emph{value[n]} and \emph{weight[n]}.\par
1) If we put it the bag, then the other items should have a total weight of no more than \emph{W - weight[n]} kg. And the total value of the bag is the value of this item plus the total value of the other items.\par
2) If we do not put it in the bag, then the other items should have a total weight of no more than \emph{W} kg. And the total value of the bag is the total value of the other items.\par
Therefore, for the first condition, the problem changes from $V[n, w]$ to $V[n - 1, w - weight[n]]$, since the other items should be chosen from the first \emph{n - 1} elements in the array; for the second condition, the problem changes from $V[n, w]$ to $V[n - 1, w]$. For both, the scale of the problem has decreased. Therefore, use some wishful thinking, we believe this problem can be solved. Hence, the recurrence function is proven to be correct.\par
Furthermore, the initial condition should be addressed. When $n = 0$, that means there is no item to choose, so \emph{V[0, w]} should always be 0; when $w = 0$, the capacity of the bag is 0. In other words, the bag cannot contain any item, so \emph{V[n, 0]} should always be 0. Additionally, it should return 0 when $weight[n] > w$ as well. This is because the bag cannot hold another item into it at this time.

%------------------------------------------------
\section{Implementation}
Please see Knapsack.java attached for a java implementation.


%------------------------------------------------
\begin{center}
--- End ---
\end{center}\par

%----------------------------------------------------------------------------
%	End
%----------------------------------------------------------------------------

\end{document}